\chapter{Resultados e Discussão}\label{cap:resultados}

Após entender o funcionamento do algoritmo de Viola-Jones e definir a metodologia para análise, foram executadas diversas iterações de classificação sobre o conjunto de imagens, variando os parâmetros de fator de escala para e número mínimo de vizinhos e cada um dos resultados obtidos pode ser exibido como um ponto no espaço ROC para que todos sejam facilmente comparados e também se torna simples observar se algum dos resultados está posicionado acima o limite lucrativo traçado. Além das classificações utilizando o classificador de Viola-Jones, foi feita uma classificação utilizando o framework Keras (fonte ref{}) para referência. 

FOTO

Em nenhuma das tentativas de classificação foi obtido um resultado lucrativo. Aprofundando a análise do resultado que mais se aproximou do cenário lucrativo, pode-se observar a matriz de confusão ref{} e é possível perceber que apenas duas imagens de faces não foram reconhecidas, entre as 17000 testadas.
Comparando tal resultado com a análise da equação que define o limiar lucrativo da classificação, é possível perceber que cada imagem positiva que é classificada como negativa gera um prejuízo muito elevado, que precisa ser compensado com a classificação correta de uma grande quantidade de imagens negativas. Para o classificador estudado, se torna extremamente difícil atingir tal resultado, mesmo com grande aumento da sensibilidade, classificando quase todas imagens como positivas.

Como uma tentativa extra e com objetivo de comparação, foi utilizado um detector de pele, que se baseia nas cores da imagem, de forma que fosse calculada a quantidade de pele existente em cada imagem, os resultados foram ordenados e então foi traçada a curva ROC \ref{} que representa os possíveis limiares da quantidade de pele existente em uma imagem que faria a mesma ser considera como face.

IMAGEM

A hipótese de que um detector de pele traria bons resultados está relacionada a necessidade de obter um número muito baixo de falsos negativos e a altíssima sensibilidade que é possível obter com tal detector, apesar disso, a quantidade elevada de falsos positivos prejudica bastante os resultados, obtendo um classificador quase igual ao aleatório. 

\begin{table}[htbp]
    \caption{Matriz de confusão com os resultados do primeiro teste}
    \label{tab:matriz_de_confusao_teste1}
    \centering
    \begin{tabular}{llrlrrr}
        Letra & Método & Fator de Escala & Nº Mínimo de VizinhosSensitividade & Especificidade & Acurácia \\
        \midrule
             A &  OpenCV &    1.30 &               5 &                            0.866257 &        0.985497 &  0.890105 \\
             B &  OpenCV &    1.30 &               2 &                            0.933450 &        0.909006 &  0.928561 \\
             C &  OpenCV &    1.05 &               0 &                            0.998480 &        0.129123 &  0.824608 \\
             D &  OpenCV &    1.05 &               1 &                            0.994795 &        0.390877 &  0.874012 \\
             E &  OpenCV &    1.05 &               2 &                            0.990175 &        0.561871 &  0.904515 \\
             F &  OpenCV &    1.05 &               3 &                            0.983509 &        0.668304 &  0.920468 \\
             G &  OpenCV &    1.05 &               5 &                            0.972690 &        0.792281 &  0.936608 \\
             H &   Keras &         &                 &                            0.936023 &        0.988538 &  0.946526 \\
        \end{tabular}
\end{table}

\begin{tabular}{llrlrrr}
Letra & Método & Fator de Escala & Nº Mínimo de VizinhosSensitividade & Especificidade & Acurácia \\
\midrule
     A &  OpenCV &    1.30 &               5 &                            0.866257 &        0.985497 &  0.890105 \\
     B &  OpenCV &    1.30 &               2 &                            0.933450 &        0.909006 &  0.928561 \\
     C &  OpenCV &    1.05 &               0 &                            0.998480 &        0.129123 &  0.824608 \\
     D &  OpenCV &    1.05 &               1 &                            0.994795 &        0.390877 &  0.874012 \\
     E &  OpenCV &    1.05 &               2 &                            0.990175 &        0.561871 &  0.904515 \\
     F &  OpenCV &    1.05 &               3 &                            0.983509 &        0.668304 &  0.920468 \\
     G &  OpenCV &    1.05 &               5 &                            0.972690 &        0.792281 &  0.936608 \\
     H &   Keras &         &                 &                            0.936023 &        0.988538 &  0.946526 \\
\end{tabular}
