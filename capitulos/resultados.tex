\chapter{Resultados e Discussão}\label{cap:resultados}

Após entender o funcionamento do algoritmo de Viola-Jones e definir a metodologia para análise, foram executadas duas iterações de detecção sobre o conjunto de imagens, com diferentes parâmetros, na primeira foram ajustados os parâmetros de fator de escala para 1.3 e número mínimo de vizinhos para 5, após analisar todas imagens foi possível preencher a matriz de confusão \ref{tab:matriz_de_confusao_teste1} e calcular a respectiva sensitividade $P(B|A) = 98.5872\%$ e especificidade $P(\overline{B} | \overline{A}) = 86.6024\%$.

\begin{table}[htbp]
    \caption{Matriz de confusão com os resultados do primeiro teste}
    \label{tab:matriz_de_confusao_teste1}
    \centering
    \begin{tabular}{cccc}\hline\hline
        & \textit{B} & $\overline{B}$ & Soma\\
    \textit{A} & 49.2936\% & 00.7064\% & 50\% \\
    $\overline{A}$ & 06.6988\% & 43.3012\% & 50\% \\
    Soma & 55.9924\% & 44.0076\% & 100\% \\
    \hline\hline
    \end{tabular}
\end{table}

Na segunda iteração, ajustando o fator de escala para 1.05 e o número mínimo de vizinhos para 3, foram obtidos os resultados apresentados na tabela \ref{tab:matriz_de_confusao_teste2} e a respectiva sensitividade $P(B|A) = 66.3923\%$ e especificidade $P(\overline{B} | \overline{A}) = 98.3479\%$.

\begin{table}[htbp]
    \caption{Matriz de confusão com os resultados do segundo teste}
    \label{tab:matriz_de_confusao_teste2}
    \centering
    \begin{tabular}{cccc}\hline\hline
        & \textit{B} & $\overline{B}$ & Soma\\
    \textit{A}& 33.1962\% & 16.8038\% & 50\% \\
    $\overline{A}$& 00.8260\% & 49.1740\% & 50\% \\
    Soma& 34.0222\% & 65.9778\% & 100\% \\
    \hline\hline
    \end{tabular}
\end{table}

Observando as duas iterações, o primeiro ponto de destaque é o impacto dos parâmetros nos resultados, comparando as medidas de \textit{B} na primeira e segunda iterações, percebe-se que a redução dos valores ajustados tanto nos parâmetros de fator de escala e número de vizinhos, fez com que mais faces forem detectadas, conforme o esperado, consequentemente, ouve um grande aumento no número de falsos negativos, onde o algoritmo encontra uma face apesar da mesma não existir. 

Analisando a sensitividade, percebe-se que na segunda iteração essa foi bastante reduzida, ou seja, dado o grupo de imagens que não possuíam nenhuma face, na primeira iteração 98.5872\% delas foram classificadas corretamente como imagens sem nenhuma face, já na segunda iteração somente 66.3923\% foram classificadas corretamente.

Já analisando a especificidade, o valor teve uma redução na segunda iteração, mas não foi uma variação tão significativa quanto a variação da sensitividade, este dado pode ser entendido da seguinte forma, dado o grupo de imagens que possuíam uma face, na primeira iteração 86.6024\% delas foram classificadas corretamente como imagens com ao menos uma face e na segunda iteração 98.3479\% foram classificadas corretamente.

Vale enfatizar que ambas análises sensitividade e especificidade estão de acordo com os resultados esperados devido aos parâmetros utilizados que fazem com que a segunda iteração detecte uma quantidade maior de faces.

Outro ponto a ser destacado é a porcentagem de imagens classificadas corretamente, ou seja, a soma das porcentagens de verdadeiros positivos e verdadeiros negativos, que na primeira iteração foi igual a 92.5948\% e na segunda iteração foi igual a 82.3702\%, esses valores demonstram que a ajuste de parâmetros não influencia apenas no limite de decisão observado na figura \ref{fig:norm_dist}, mas também nas distribuições.
